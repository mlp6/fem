\documentclass[12pt]{article}

\usepackage{epsfig}  % packages are external code that gives you access to
                     % commands that others have written
                     % this particular package allows you to put PostScript
                     % images into your document

\begin{document}
\section*{Hypermesh Tutorial}
Rectilinear Mesh Generation Using Hypermesh

\section*{Set Global Properties to LS Dyna}
\begin{enumerate}
\item Click on the global button in the lower-right.
\item Click the load button beside the template file. Select latest version of LS Dyna key.
\item Click return.
\end{enumerate}
\section*{Define the Part and Set Material Properties}
\vspace*{10Pt}
\noindent
\begin{enumerate}
\item Goto the Collectors Menu.
\item Click on the up arrow beside collector type.
\item Select Mat to define the collector as a material.
\item Enter a name to describe the part
\item Click on card image. Select the desired material card. Use MATL-TH1
for a thermally isotropic material.
\item Click Create/Edit
\item Enter the properties for the material. For thermal properties, TRO is 
the material density, TGRLC and TGMULT should be zero, HC is the heat capacity,
and TC is the thermal conductivity for a given material.
\item Return to the main menus.
\end{enumerate}


\section*{Uniform Mesh Generation}
\vspace*{10Pt}
\noindent
\begin{enumerate}
\item Click on Perf rather than Std besides GFX. This will make mesh generation much faster.
\item Goto Geometry, then Create Nodes.Enter four nodes to define one face of your mesh.
For uniform meshes without biasing, it is suggested to define nodes along the x-y plane
\item Goto 2-D and select Planes. 
\item Select Nodes in the left drop down menu. Click on Nodes and highlight the nodes 
you are interested in defining a plane for.
\item Select surface-only in the right drop down menu. 
\item Select trimmed and then click on Create.
\item Go back to the 2-D menu and select Automesh.
\item Click on the surface. It should turn from red to white.
\item Click on mesh, making sure interactive mesh is highlighted.
\item Numbers will now appear along the sides of the surface. Selecting these allow defining
the number of elements along that edge.Alternatively, enter the desired element density into
elem density and click Set Edge To.
\item Define element densities for all four edges.
\item Click Mesh to generate the mesh. Click Reject to reject a generated mesh.
\item Return to the main menu once a suitable mesh has been generated.
\item The two-dimenional mesh now needs to be dragged to generate a 3-D volume. For example,
a planar mesh can be generated at the face of the transducer and then grown along the z axis
to make the model volume.
\item Goto 3-D. Goto Drag.
\item Click drag elems.
\item Click elements. Select all. 
\item Click on the arrow button beside N1 N2 N3.
\item Select the direction to drag the mesh in under to create a volume. For a mesh defined
in the x-y plane, the z direction should be specified. 
\item Enter the distance to drag the mesh.
\item On Drag defines the element density in the dragged direction. Enter the desired element
density.
\item Select Drag+ to drag elements in the + direction. Select Drag - for the reverse.
\item To view the mesh, click on View, then select Iso1.
\item Click return.
\item Sections with different densities can be generated in a similar fashion. First create a plane
divided into subsections depending on the number of desired densities. Use Automesh to specify the desired
number of edge elements for each section. Create the planar mesh and drag this new mesh to create a volume. 
\end{enumerate}

\section*{Removing unneeded nodes and surfaces}
\vspace*{10Pt}
\noindent
\begin{enumerate}
\item The volume is almost finished. Now the nodes and surfaces created earlier should be
deleted. To do this, click on Display and turn Elems off. Only the initial nodes and surfaces
will now be displayed.
\item Goto Geom
\item Goto Temp Nodes
\item Click nodes and select all.
\item Select clear.
\item To remove the surface, goto Tools and then Delete.
\item Select Surfs and click on the surfaces to delete. Select Delete entity.
\item To check the number of elements and nodes, goto Tools and then Count. Select all.
\item Your mesh is almost finished. To write to a ls-dyna file, goto Files and then select Export.
\item Make sure the dyna key is selected and click write as. Enter a file name and click Save.
\item Open the saved .dyn file in your favorite text editor. Search for Shell. This section of the dyna
deck defines shell elements generated from the 2-D automesh command. We only want solid elements.
Delete all the nodes under Shell elements.
\item Enter the desired control cards into deck.
\item Drink a beer. Watch a movie. Work from home. Your mesh is now finished.
\end{enumerate}
\end{document} 